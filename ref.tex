
\documentclass[a4paper,12pt]{report}
\usepackage[14pt]{extsizes} % для того чтобы задать нестандартный 14-
\usepackage{mathtext}
\usepackage[TS1,T2A]{fontenc}
\usepackage[utf8]{inputenc}
\usepackage[english,russian]{babel}
\usepackage{longtable}
\usepackage[left=2cm,right=2cm,
top=2cm,bottom=2cm,bindingoffset=0cm]{geometry}
\usepackage[final]{pdfpages}  
\usepackage{color}
\usepackage{wasysym}
\usepackage{graphicx}
\usepackage{indentfirst}

\graphicspath{{pictures/}}

\usepackage{tocloft}
\renewcommand{\cfttoctitlefont}{\hspace{0.38\textwidth} \bfseries\MakeUppercase}

\renewcommand{\cftbeforetoctitleskip}{-1em}
\renewcommand{\cftaftertoctitle}{\mbox{}\hfill \\ \mbox{}\hfill{\footnotesize Стр.}\vspace{-2.5em}}
\renewcommand{\cftsecfont}{\hspace{3pt}}
\renewcommand{\cftsubsecfont}{\hspace{1pt}}

\renewcommand{\cftparskip}{-1mm}
\renewcommand{\cftdotsep}{1}



\linespread{1.15} % полуторный интервал
\renewcommand{\rmdefault}{ftm} % Times New Roman
\frenchspacing

\usepackage{titlesec}
\renewcommand{\contentsname}{Содержание}
\titleformat{\chapter}[display]{\filcenter}{\bfseries{\chaptertitlename} \thechapter}{8pt}{}{}

% Настройка вертикальных и горизонтальных отступов
\titlespacing*{\chapter}{0pt}{-30pt}{8pt}
\titlespacing*{\section}{\parindent}{*4}{*4}

\setcounter{page}{1}
\addtocontents{toc}{\protect\thispagestyle{empty}}
\addtocontents{toc}{\protect\pagestyle{empty}}
\author{}

\begin{document}
	
	\title{}
\pagestyle{empty}
	\begin{center}
	 \textbf{МИНИСТЕРСТВО ОБРАЗОВАНИЯ И НАУКИ РОССИЙСКОЙ ФЕДЕРАЦИИ}\\
 федеральное государственное образовательное учреждение\\ высшего образования\\
«СИБИРСКИЙ ФЕДЕРАЛЬНЫЙ УНИВЕРСИТЕТ»\\
(СФУ)\\
\vspace{0.2em}




\end{center}

\vspace{6em}
\begin{center}
	 \bf{ РЕФЕРАТ } \\% \\ означает перенос
	\vspace{1em} по дисциплине «Теория систем»\\
\vspace{0.5em} на тему \\
\vspace{0.5em} 
\large{Особенности моделей диалектической логики}\\
\end{center}
\vspace{0.5em}
\textsl{}
\vspace{6em}




\begin{flushright}
Выполнил:\\
студент ЗКИ21-16БВВ \\
Хорошко Е.М.\\
\vspace{1.5em}
Проверил:\\
доцент кафедры ИСУ, к.т.н.\\
Иконников О. А.\\
\end{flushright} 
\begin{center}
	\vspace{8em}
	Красноярск 2024
\end{center}

\newpage

\pagestyle{empty} 
\def\contentsname{Содержание}
\tableofcontents

\newpage

\chapter*{ВВЕДЕНИЕ}
\addcontentsline{toc}{chapter}{ВВЕДЕНИЕ} \\
\pagestyle{plain} \\

В современном мире уже многие рутинные операции автоматизированы, автоматизируются процессы проектирования, управления предприятиями и организациями. Практика потребовала формализации и автоматизации более сложной творческой деятельности, связанной с принятием решений в условиях неопределенности, размытости и даже противоречивых исходных данных.  Проблемы формализации творческой деятельности, «оцифровка» логики «здавого смысла» становятся в один ряд с проблемами сложности и нелинейности окружающего мира, которые в 21 веке игнорировать уже невозможно. Неизбежно происходит слияние точных наук и наук гуманитарного спектра, и даже искусства. В рамках этого слияния происходит синтез двух дополняющих друг друга способа познания природы: аналитического и интуитивного, «...между неформальным, образным мышлением человека и формальными моделями классической математики сложился как бы «спектр» методов, которые помогают получать и уточнять (формализовать) вербальное описание проблемной ситуации, с одной стороны, и интерпретировать формальные модели, связывать их с реальной действительностью, с другой. » (стр.12 Денисов А.А.) \\

В этих условиях размытости актуальным становится философский анализ. «Философия является не только картиной материальной и духовной действительности (теорией, взглядом, учением, концепцией), но также методом познания этой действительности...»  (стр. 386 Алексеев учебник) Категориальный аппарат философии  все более глубоко проникает в частные науки, в ткань всего научного знания, осуществляя синтез знания на эмпирическом и теоретическом уровнях. Логико-гносеологические модели философии находит свое применение для развития общенаучных методов познания, например, системного подхода, метода моделирования. Будучи примененными к построению научных теорий, принципы диалектики как логики включаются в состав их логических оснований.  Диалектическая логика может дать универсальный аппарат творческой деятельности.
Классическая формальная логика и логика диалектическая объединяются в методах и подходах системного анализа явлений и процессов.\\
 Одним из таких методов является рассматриваемый в работе информационный подход. Проблема системного моделирования объектов и ситуаций с неопределенностью заключается в столкновении взаимно исключающих друг друга требований точности и обозримости. В информационном подходе предлагается способ снятия такого противоречия, представив данную проблему в виде дискретной модели непрерывного бытия. Предложенная логико-гносеологическая модель основана на формализации диалектичской логики. Остановимся подробнее на целях и задачах данной работы:\\

	Цель — ознакомиться с основными понятиями информационного подхода, рассмотреть способы измерения сложности систем различной природы, рассмотреть особенности применения логико-гнеосеологического комплекса понятий и законов диалектической логики к построению моделей систем различной сложности.\\
	
	Задачи —  рассмотреть основные понятия и законы диалектической и формальной логики,  обозначить основные понятия информационного подхода, сформулированного А.А.Денисовым в теории информационного поля, рассмотреть принципы моделирования, основанные на формализации диалектической логики, раскрыть методологическую роль философии в  вопросах моделирования сложных систем и информационных процессов на примере информационного подхода. \\
	
	Объектами исследования являются информационный подход как специальный метод системного анализа, а так же законы диалектической и формальной логики.\\
	
	Предметом исследования будет являться применение диалектической логики как логико-гносеологического метода к моделированию сложных систем и процессов принятия решений, связь теории отражения и понятий чувственной и логической информации, диалектика абсолютной и относительной истины, континуальность и дискретность информации как парной категории к материи,  отношения диалектической и формальной логики к моделированию сложных систем, статика, кинематика и динамика моделей диалектической логики,  закономерности диалектического мышления, представленные в формализованом виде.\\
	
	В первой главе данного реферата дается  определение диалектики, обозначаются основные понятия и законы диалектической логики, а также представлено краткое описание законов классической формальной логики.  Вторая глава посвещанна описанию основных понятий информационного подхода, как метода системного анализа, основанного на теории информационного поля А.А. Денисова. Предложенная А. А. Денисовым теория основана на общности явлений и процессов в системах различной физической природы. В третей главе показано отношение между логиками формальной и диалектической, дано определение информации, дающее возможность снятия диалектического  противоречия между формальным (абстрактным, математически строгим) мышлением, отражающимся в законах классической формальной логики и диалектическом мышлением, на примере чувственной информации представлен способ символизации диалектического закона отрицания отрицания, дискретное прдставление непрерывного бытия, раскрыто понятие прагматической информации цель = смысл, системная сложность и закон целостности системы. В четвертой главе приводится список формализованных законов диалектической логики в рамках информационного подхода. В заключение подводится краткоий итог проделанной работы, представлены основные выводы.  
\\

	Теоретическую основу исследования составили работы ведущих отечественных ученых в области теории систем и системного анализа A.A. Денисова В.Н., Волковой, Козлова Н, в области философии и логики  – Алексеева, Спиркина, Виноградова, Фролова, А.Н. Троепольского. В области кибернетики - Тюхтина\\



\chapter*{1.  Диалектика и диалектическая логика. Основные понятия. }
\addcontentsline{toc}{chapter}{1.  Диалектика и диалектическая логика. Основные понятия. } \\

Понятие диалектики (от греч. Dialektike techne – искусство вести беседу, рассуждать) употреблялось в истории философии в разных значениях. Родилось оно в древнегреческой культуе. Под диалектикой понимали искусство спора, дискуссий, умение плодотворно развивать обсуждаемую тему. Уже в философском учении Сократа практика общественной полемики рассматривалась как искусство обнаружения истины путем столкновения и согласования различных и даже противоположных мнений. В результате полемики осуществлялся переход от частных случаев понятий к искомым общим их определениям. Ученик Сократа, Платон, представил диалектику как метод анализа и синтеза понятий, как движение мысли от многообразных конкретных значений к общим понятиям – идеям, выражавшим истинно сущее. [стр 96-97  Введение в философию: Учебник для вузов. В 2ч. Ч. 2./ Фролов И.Т., Араб-Оглы Э.А., Арефьева Г.С. и др. - М.: Политиздат, 1989. - 639 с.] \\

	Со временем было осознано, что приемы столкновения мнений и диалектического разрешения противоречий применимы не только в ситуациях живого спора реальных людей. Они важны также при анализе противоборствующих взглядов, позиций, направлений мысли. Диалектика – способ рассуждения, при котором противоположные позиции не перечеркивают, а дополняют, обогащают, уравновешивают одна другую. \\

	В виде стройной теоретической системы диалектика впервые предстала в философии Гегеля. В его работах сформировался понятийный аппарат диалектики и были сформулированы закономерности, отражающие универсальные связи мира и познания. Диалектика предстала в виде знания о диалектических закономерностях – универсальной связи явлений, единства противоположностей и развитие через диалектические отрицания.  [стр. 102, 103 – Введение в философию: Учебник для вузов. В 2ч. Ч. 2./ Фролов И.Т., Араб-Оглы Э.А., Арефьева Г.С. и др. - М.: Политиздат, 1989. - 639 с.]\\

	В современной диалектической логике как методологии познания различают общие и специфические законы диалектики. Первые представляют собой законы развития природы, общества и человеческого познания. К ним относятся законы:\\
	- перехода количественных изменений в качественные, \\
	- единства и борьбы противоположностей,\\
	-  отрицания отрицания [6, c. 43 6. Краткий словарь по логике. М., 1991.].\\
	 Специфическими законами диалектики являются те, которые присущи лишь сфере познания [6, c. 43 6. Краткий словарь по логике. М., 1991.]. Это закономерности соотношения абсолютной и относительной истины, восхождения от абстрактного знания к конкретному, соотношения анализа и синтеза, индукции и дедукции, принцип конкретности истины. [6, c. 43—44  Краткий словарь по логике. М., 1991.]. Центральная роль  в диалектической логике отводится закону единства и борьбы противоположностей и его ядру — диалектическому противоречию, которое   “...определяется как единство противоположных характеристик А и В, принадлежащих некоторым объектам. Эти характеристики одновременно исключают и предполагают друг друга, что обеспечивает целостность объекта, их борьба приводит к его гибели. Диалектические противоречия являются источником изменения и развития объекта, его «самодвижения»“. (А.Н. Троепольский История противостояния формальной и диалектической логики) \\

В свою очередь формальная логика рассматривает четыре закона логического мышления: закон тождества, закон противоречия, закон исключенного третьего и закон достаточного основания. Эти законы выражают коренные черты мышления: \\
- определенность, \\
- непротиворечивость, \\
- последовательность, \\
- обоснованность.\\
Закон тождества: В данном рассуждении, споре, дискусии каждое понятие должно употребляться в одном и том же смысле.\\
Закон противоречия: Два противоположных высказывания не могут быть оба истинными в одно и то же время, в одном и том же отношении.
Закон исключенного третьего: Из двух противоречивых суждений всегда одно истинное, другое ложное, а третьего быть не может.
Закон достаточного основания: Всякая истинная мысль должна быть обоснованной. [стр 85 Виноградов С.Н., Кузьмин А.Ф. Логика: учебник для средней шкрлы – М.: Издательство “Наше завтра”, 2022. -176 с. ]\\
	Если формальная логика есть наука о законах и формах отражения в мышлении постоянства, покоя в объективном мире, то специфика диалектической логики — изучение отражения в законах и формах мышления процессов развития, внутренних противоречий явлений, их качественного изменения, перехода одного в другое и т. д.  Диалектические законы - это наиболее общие законы развития природы, общества и мышления.\\ Диалектическая логика не отвергает формальную логику, а позволяет уточнить её границы, место и роль в изучении законов и форм мышления. \\
Диалектика задает общие ориентиры познавательной деятельности в области теоретического естествознания, а разработка диалектико-логических принципов познания, проводимая в тесном единстве с обобщением новейших достижений методологии естественных наук, привносит научное развитие самой философии и придает ей практическую значимость. (Алексеев) \\


 \chapter*{2. Информационный подход }
 \addcontentsline{toc}{chapter}{2. Информационный подход } \\

К концу XX века был открыт ряд законов сохранения (материи и движения; полной энергии; полного импульса; полного момента количества движения; тяжелых частиц; электрического заряда), которые, с одной стороны, отражают разнообразие природы, а, с другой – общность (все они являются законами сохранения материи и ее свойств), сходство законов в средах различной физической природы. Сопоставление закономерностей показало явное внешнее формальное сходство уровнений для механических, электрических, гидравлических явлений и процессов, а также газодинамики.  Это привело к возникновению интегральных концепциий для исследования процессов в системах различной физической природы. \\
	В 1960—1970-е гг. на основе обобщения аналогий А.А.Денисов предложил и развил теорию для совместного исследования гидравлических, пневматических и электрических процессов автоматики, названную электрофлюидикой. На основе данной теории была создана техника непосредственной связи по управлению, электрогидравлические и электропневматические устройства, включая широкий спектр преобразователей рода энергии сигналов и ЭГД генераторов высокого напряжения для скоростных летательных аппаратов, управляющие устройства и преобразователи скорости жидкостей и газов, которые отличает полное отсутствие подвижных механических деталей и узлов, что делает их нечувствительными к большим ускорениям и радиации.  Разработанные А.А.Денисовым и его учениками устройства использовались в работах Центрального аэрогидродинамического
института им. Н. Е. Жуковского, Института проблем управления Академии наук СССР и ряда организаций оборонного комплекса. Одной из общеизвестных областей применения основанных на электрофлюидике технологий являются струйные принтеры. Способ струйной печати, разработанный и предложенный в СССР А. А. Денисовым, в 1980-е годы опережал соответствующие технологии западных стран, в том числе предлагавшиеся в то время фирмой Philips.\\
	В последующем развитии этой теории А.А. Денисовым была предложена теория информационного поля, ставшая основой информационного подхода к анализу систем. Эта теория позволяет с единых позиций описывать процессы в различных системах − технических, организационных, социальных, включая анализ процессов управления общественными конгломератами (экономика, политика, наука, образование и т. п.) Основу теории А.А.Денисова составляет математическая теория поля и формализованное представление диалектической логики. Информационный подход – это дискретный вариант теории информационнго поля, представляющий метод системного анализа.\\
	В основе информационного подхода лежит философско-гносеологическая модель мира, основанная на диалектических законах отражения и понятии информации как парной философской категории к понятию материи, определяющей ее структуру. Информация выступает в триаде – информация для нас, информация в себе и практическая информация.\\
Опираясь на философию И. Канта, который вводит понятия «вещь в себе» и «вещь для нас» и высказывает предположение о существовании чувственности и рассудка, на работы Ф. Энгельса, который вводит понятия чувственного и логического познания, К. Маркса и В.И. Ленина, в чьих трудах сформировались основные положения теории познания как теории отражения, А.А. Денисов определяет основные формы существования информации в виде чувственной информации или «информации восприятия» J, логической информаци, называемой «информационным потенциалом» или сущностью воспринимаемой информации H, и их логического пересечения, названного в рассматриваемой теории «информационной сложностью» (содержанием, смыслом). [Волкова] \\
	Чувственная информация обозначается $J = RM$, где $М $– материальное свойство, а R - безразмерная константа, характеризующая логическую реакцию (поведение) отражающего объекта (субъекта) на поток существования (отражения). Между чувственной и логической информациями есть различие количественное, ибо информация в себе $J_c$ в общем случае больше информации для нас $J_н$. В линейном приближении: \\
\begin{equation}
\label{trivial}
	 J_н = R_k \cdot J_c  = R_k \cdot M,                                          				          
\end{equation}
где $M$ − измеряемое материальное свойство (масса, цвет, заряд и т.п.), создающее $J_c$; $J_н$ − чувственная информация (информация для нас) или информация восприятия, которую в дальнейшем автор теории будет использовать без индекса; Rk − относительная информационная проницаемость среды .
	Сумма потоков информации от отдельных частей материального объекта или от совокупности материальных объектов формирует информационное поле вокруг воспринимающего его. Если говорить об отражении материального объекта или поля некой произвольной замкнутой вокруг него поверхностью, то полную информацию можно формализовано представить как сумму потоков информации, приходящихся на единицу dS площади этой поверхности, т. е. из $O = dJ/dS$. Далее А.А. Денисов использует теорему Гаусса, являющуюся, по мнению автора теории, математическим выражением философского положения о познаваемости мира:

\begin{equation}
\label{trivial}
	M = \oint\limits_{S}O dS \mbox{ или }	   J_c = \oint\limits_{S}O dS ,	      
\end{equation}   							
       	         
где $O$ − вектор интенсивности потока существования (отражения). Интеграл берется по замкнутой поверхности $S$, охватывающей изучаемое явление или объект. С учетом (2) теорему Гаусса можно представить в форме:
\begin{equation}
\label{trivial}
	 J_н = \oint\limits_{S} R_k O dS = \oint\limits_{S} O_н dS, 								
\end{equation} 
где $O_н = R_k \cdot O$ − вектор интенсивности отражения; $R_k$ − информационная проницаемость среды.\\

	Логическая информация (сущность) характеризует целый класс однородных в определенном отношении объектов, являясь семантическим синтезом законов логики, правил функционирования системы и ее элементов, образующих функционал ее существования 
\begin{equation}
\label{trivial}
R \cdot M = \oint\limits_{S} E dS,
\end{equation} 
	 									
где $E$ − вектор напряженности поля логики; $R = R_k \cdot R_o$ − безразмерная константа, харктеризующая логическую реакцию (поведение) отражающего объекта (субъекта) на поток $O$. Тогда закон логического отражения, олицетворяющий адекватность отражения в отсутствие априорного знания, можно записать следующим образом:

	\begin{equation}
\label{trivial}
E = O R(O),
\end{equation} 

где $E$ − вектор интенсивности логики (напряженности поля логики). В общем случае $Ro$ зависит от $O$, но всегда 

	\begin{equation}
\label{trivial}
E  = - grad H,
\end{equation} 

где $Н$ – потенциал поля (сущность воспринимаемой информации).\\

	Логическая информация зависит от чувственного отражения. Информационная сложность или содержание (смысл) C определяется пересечением ( логическим произведением, а в частных случаях − декартовым произведением )$J$ и $H$:

	\begin{equation}
\label{trivial}
 С = J \cap  H \mbox{ или } C = J \times H.		
\end{equation}						
	Для конструктивного использования понятий чувственная и логическая информация вводятся соответствующие детерминированные и вероятностные меры. Способ опосредования (усреднения) J может быть различным, для чего вводится параметр γ, который может выбирать постановщик задачи. Тогда:
	\begin{equation}
\label{trivial}
 H = \sqrt[ \gamma ]{ \frac{1}{n} \sum_{i = 1}^n J_i^\gamma}  , 
\end{equation}										 

где $J-i$ − результаты измерения $A_i$; $n$ − объем понятия, т.е. число, охватываемых понятием объектов; $\gamma$ − параметр логики усреднения, при различных значениях которого получаются различные выражения для определения $H$, приведенные в таблице 1.
	Логическую информацию $H$ можно определить через вероятностные характеристики. Если учесть, что, $H$ характеризует не единичный объект, а класс однородных в определенном смысле объектов или свойств, то $H$ можно определить через плотность вероятности $f(J_i)$ того, что $J$ имеет значение $J_i$:

	\begin{equation}
\label{trivial}
H = \int f( J_i)dJ_i  .
\end{equation} 										

	В частном случае вместо плотности вероятности можно охарактеризовать класс однородных объектов просто вероятностью qi и представить $J_i$ в логарифмической форме. Тогда получим:

	\begin{equation}
\label{trivial}
H =  -\sum_{i = 1}^{n}q_i \log p_i
\end{equation} 

Значения qi и pi могут быть не равны, но возможны ситуации, когда $q_i = p_i$, что имеет место в формуле К . Шеннона:

	\begin{equation}
\label{trivial}
H =  -\sum_{i = 1}^{n}p_i \log p_i.
\end{equation} 									      

	Для практических приложений вводится прагматическая (целевая) информация Hц, которая описывается моделью, аналогичной (9), но вероятность недостижения цели $p_i$ заменена на сопряженную ($1 - p_i'$):

	\begin{equation}
\label{trivial}
H_ц = - \sum_{i=1}^{ n} p_i \log (1 - p_i'),
\end{equation} 
где $p_i'$ − вероятность достижения цели; $q_i$ − вероятность того, что оцениваемая компонента может быть реализована и/или использована для достижения цели.\\

Для измерения чувственной информации используют следую-
щие меры:\\
• при детерминированном способе измерения информации J вво-
дится мера отраженной в нашем сознании элементной базы сис-
темы в форме J = A/ΔA, где А – общее количество каких-либо
знаков событий, воспринимаемых измерительными прибора-
ми или нашими органами чувств; ΔA − «квант», с точностью
до которого нас интересует воспринимаемая информация, или
разрешающая способность прибора;\\
• при вероятностном способе измерения принята логарифмиче-
ская мера J = –log 2 p i, где p i − вероятности события. В случае когда J используется для достижения цели, p i − вероятность недостижения цели, т. е. степень «целенесоответствия».\\
В обоих случаях принята единица измерений «бит», поскольку
J = A/ΔA = –log2 0,5 j = –log2 0,5A/ΔA = –log 2 p, где p = 0,5A/ΔA − совместимая вероятность А/
ΔA событий, априорная вероятность каждого из
которых равна 0,5; тогда минимальное значение (единица измерения)
∆J = –log2 0,5 = 1 бит, где 0,5 − вероятность наличия или отсутствия
минимального значения информации.\\
В качестве единицы измерения можно принять другие логиче-
ские шкалы. Например, если log 8 , то единица измерения «байт»; если
принят натуральный логарифм, т. е. основание е = 2,7, то «непер»
(Нп) и т. д.\\
Основные сособы оценки чувственной и логической информации приведены в сопоставительной таблице 1:\\

\textit{Таблица 1 - Способы оценки $J$ и $H$}

\begin{tabular}{|p{3.5cm}|p{4cm}|p{3.5cm}|p{3.7cm}|}
\hline
\multicolumn{4}{|c|}{\textbf{Меры информации}}\\
\hline
 \multicolumn{2}{|p{7.5cm}|}{ Информация восприятия (чувственная информация) $J$ (элементная база)}&
 \multicolumn{2}{p{7.5cm}|}{ Логическая информация (информационный потенциал) (сущность)}\\
 \hline
Детерминированный способ измерения & 
 $$= A_i/\Delta A_i$$
где $A_i$ – значение измеряемой величины;
$\Delta A_i$  - «квант», с точночтью до которого лицо, принимающее решение (ЛПР), интересует воспринимаемая инормация (единица измерения, разрешающая способность прибора) & 
$$H = \sqrt[ \gamma ]{ \frac{1}{n} \sum_{i = 1}^n J_i^\gamma} $$
где $J_i$ – результаты измерения $A_i$;
$n$ – объем понятия об охватываемых измерением объектах;
$\gamma$ – параметр усреднения. & 

 При $\gamma = 1$ получим среднее арифметическое
$$H = { \frac{1}{n} \sum_{i = 1}^n J_i^\gamma} $$
 При  $\gamma = 0$ получим среднее геометрическое
$$H = \sqrt[ \gamma ]{ \prod_{i = 1}^n J_i} $$
При $\gamma = –1$ ‒ среднее гармоническое
$$H = n\ \sum_{i = 1}^n \frac{1}{J_i^} $$\\
\hline

\bf{Вероятностный способ измерения} &

$$J_i =-\log_2(p_i) ,$$
  где $p_i$ – вероятность события. 
В случае использования информации для достижения цели pi называют вероятностью недостижения цели или степенью нецелесообразности. &
 \multicolumn{2}{p{7.2cm}|}{ 
 $$ H = \int f(J_i)dJ_i \Rightarrow   $$ 
 $$H = \sum_{i = 1}^{n} q_i J_i =  -\sum_{i = 1}^{n}q_i \log p_i $$
где $q_i$ – вероятность использования элемента информации.
При $q_i = p_i$ . 
При равновероятном выборе элемента
 $$p_i = 1/n \mbox{ и } H = - \sum_{i = 1}^{ n } {\frac{1}{n}} \log \frac{1}{n} = \log n .$$
Для прагматической информации 
$$H_ц = - \sum_{ i = 1 }^{ n } q_i \log(1-p_i'),$$
где $p_i'$- вероятность достижения цели, степень целесоответствия
}\\
\hline

\end{tabular}\\
\\


	Таким образом, $J$ и $H$ могут измеряться различными способами − детерминированно и с помощью вероятностных характеристик. В некоторых приложениях могут быть использованы одновременно обе формы представления информационных характеристик − и детерминированная, и
вероятностная, а также − переход от одной формы к другой. Следует отметить особенности вероятностных характеристик, используемых в информационном подходе. В частном случае pi может быть статистической вероятностью, определяемой на основе репрезентативной выборки, подчиняющейся той или иной статистической закономерности. Однако в общем случае вероятность достижения цели pi' и вероятность использования оцениваемой компоненты (свойства) при принятии решения qi могут иметь более широкую трактовку и использоваться не в строгом смысле с точки зрения теории вероятностей, справедливой для стохастических, повторяющихся явлений, а характеризовать единичные явления, события, когда pi' выступает как степень целесоответствия. 	В качестве единицы измерения информации принята единица, основанная на двоичном логарифме, дающая в качестве минимальной единицы информации величину 1 бит, 	по аналогии с предшествующими исследованиями Р. Хартли, К. Шеннона, А.А. Харкевича. А.А.Денисов так же отмечает, что «в принципе могут быть приняты и иные меры сжатия информационной шкалы − восьмеричные логарифмы − байты  уже нашедшие применение для оценки объемов информации в вычислительной технике или даже, не применяющиеся пока − десятичные логарифмы (единицу можно назвать, например, «дек»), натуральные логарифмы («непер») и т. п. » [Денисов]
	В дискретном варианте теории информационного поля А.А. Денисова основой являются меры чувственной и логической информации ($J$ и $H$), а также предлагаются два способа их измерения – детерменированный и вероятностный.



\chapter*{3. Особенности моделей диалектической логики в рамках информационного подхода }
\addcontentsline{toc}{chapter}{3. Особенности моделей диалектической логики в рамках информационного подхода } \\



В основе возможностей информационных моделей лежит сближение  принципиально различных математического и философского подходов описания действительности, как двух противоположных направления движения к предельной абстракци.\\ Первое направление формирует в своем пределе метафизические объекты формальной логики, имеющие статус абсолютной истины. Абсолютная точность объектов математики позволяет применить к ним ряд столь же абсолютно точных правил преобразования, сохраняющих абсолютную точность результата. Совокупность таких правил сводится к формальной логике, ощутимо облегчаюшей рутинные операции и легко реализуемой посредством кибернетической техники.\\
 Второе направление - путь последовательных и безграничных обобщений, ведущих через понятия частных наук к всеобщим философским категориям. Характерная особенность этого направления - последовательное возрастание размытости понятий, охватывающих по мере обобщений все больший и больший объем реальных явлений и ведущее к категории материи, охватывающий весь безграничный объем данной в ощущениях объективной реальности и в силу этого бесконечно размытой во всех свойствах, кроме свойства существовать и отражаться. Философские выводы применимы с учетом принципа конкретности истины, а результат логических операций над философскими категориями будет размытым и приближенным в ввиду их относительности. Однако именно из-за размытости на всю реальность философия говорит на языке этой реальности и является наукой обо все сущем. [Волкаова гл. особенности моделей]\\ 
 Для информации,как универсальной категории, определяемой в качестве продукта отражения материи, характерна как размытость, так и безразмерность, и ее количество зависит от разрешающей способности (точности) наших органов чувств и дополняющих их измерительных приборов. Колличество информации прямо зависит от точности задания $\delta A$, и поэтому может быть различным, зависит от конкретных целей моделирования.\\
 Информация - диалектический объект в силу размытости, а также способности эволюционировать, например, увеличивать свое количество по мере совершенствования измерительных приборов, служащих средством ее получения. Одновременно с этим к информации применимы законы классической и формальной логики за исключением закона тождества и исключенного третьего, а также вытекающие из них следствий. Вместо них к информации применимы законы единства противоположнойстей и отрицания отрицания, что и позволяет учесть в рассматриваемых информационных моделях как развитие, так и метафизически несовместимые противоположные требования.\\
  Колличество информации, потенциально содержащейся в материальном объекте, всегда конечно, поскольку соответствующее материальное свойство всегда реализуется в рамках конечной точности или минимального диапазона существования $\Delta B$. Эта информация именуется потенциальной и вычисляется как $A/ \Delta B$. Безызбыточная часть потенциальной информации тем самым численно равна измеряемой материи, точнее конкретного материального свойства $ M_k = A/ \Delta B$. В то же время для объектов ествественной природы актуальная информация $\Delta A$ формируется в процессе эволюции, представляя элемент системы.\\
  Процесс становления чувственной информации рассматривается с учетом диалектики отражения и самоотражения. Материя $M_k$ воздействует на органы чувств и с учетом реальной информационной пронецаемости $R_k$ среды должна была бы отразиться ими как $J_k = R_kM_k$, однако этому препятствуют процессы самоотражения органов чувств. По мере отражения могут возникать приращения $\Delta J$ информации, которые являются новообразованиями, чуждыми предшествующей информации и отрицающими ее, поскольку эти приращения являются \textit{"неА"} по отношению к принятой за $A$ информации в предшествующий промежуток времени. Этот факт представлен выражением, где из потенциальной информации $J_o$ вычитаются приращения информации ($\Delta_1 J$ и $\Delta_2 J$), хотя и порождаются ею же. При этом приращение $\Delta_1 J$ представляет приращение информации за характерный промежуток $\tau$ времени $t$, так что накопление "$не J_k$" за время $\tau$ дает $\Delta_1 J$, т.е. 
  
\begin{equation}
\label{trivial}
 \Delta_1 J = \tau dJ_k/dt.
\end{equation} 
 
  
  Поскольку отражение и отрицание идут в общем случае с непостоянным темпом, то $\Delta_1 J$ представляет приращение $\Delta_1 J$ за характерный промежуток времени $\tau'$, так что накопление $"не\Delta_1J"$ за время $\tau'$ дает $\Delta_2 J'$, т.е.
  
 \begin{equation}
\label{trivial}
 \Delta_2 J = \tau' d\Delta_1J/dt.
\end{equation} 
Поскольку $\Delta_1 J$ само является $неA$ по отношению к $J_k$, то $\Delta_2J$ является уже $"\mbox{не } неА"$, т.е. отрицанием отрицания $J_k$, что символизируется $d/dt(dJ_k/dt) = d^2J_k/dt^2$, так что 

 \begin{equation}
\label{trivial}
 \Delta_2 J = \tau'\tau d^2J_k/dt^2 = Ld^2J_k/dt^2,
\end{equation} 
где $L = \tau'\tau $.
Процесс самоотражения образует два контура отрицательной обратной связи: один - по скорости, другой по ускорению процесса, которые замедляют процесс отражения, уменьшая в каждый момент времени актуальную информацию $J_k$ по сравнению с потенциальной информацией $J_o$, так что

\begin{equation}
\label{trivial}
 \ J_k = R_k(J_o - \Delta_1 J - \Delta_2 J) = R_k(J_o - \tau' d\Delta_1J/dt - Ld^2J_k/dt^2.
\end{equation} 

C учетом, что $J_o = M_k$ , получим соотношение, символизирующее процесс становления информации(знания) как совокупности внешнего отражения материи(первое слагаемое) и самоотражения(второе и третье слагаемые):

\begin{equation}
\label{trivial}
 \ M_k = J_k/R_k + \tau dJ_k/dt + Ld^2J_k/dt^2.
\end{equation} 
 Данное соотношение символизирует синтез знания как единство противоположностей, тезиса (первое слагаемое) и антитезиса(второе слагаемое), опосредованных переходным членом(третье слагаемое). Второе и третье слагаемые символизируют соответственно отрицание и отрицание отрицания информации. Математически это дифференциальное уравнение второго порядка, связывающее информацию и материю и позволяющее оперировать количеством того и другого. Данное уравнение может использоваться и для чисто качественного, содержательного описания диалектики отражения, поскольку эта символика может интерпретироваться и в естественном языке.\\
 Процесс становления логической информации $H$ может быть описан аналогично процессу отражения чувственной информации $J$. Так как логическая информация $H$ есть в общем случае взвешенное среднее чувственных информаций об однородных в определенном отношении объектах  
 
 \chapter*{4. Формализация диалектической логики в теории информационного поля А.А. Денисова }
 \addcontentsline{toc}{chapter}{4. Формализация диалектической логики в теории информационного поля А.А. Денисова } \\
 

	Развивая идеи теории информационного поля А.А. Денисов сформулировал систему формализованных законов диалектической логики (8) .Под формализацией можно понимать уточнение содержания изучаемых объектов и процессов с применением какой-либо устойчивой конструкции, и в частности
математического аппарата.[cnh 6 Волкова Постепенная формализация методов принятия решений]:\\
    • Первый закон диалектической логики — основной закон логики, справедлив как для классической логики, так и для диалектики. Согласно этому закону сущность H понятия обратно пропорционально его объему n. Здесь под объемом понятия подразумевается общее число однородных объектов или явлений, информация J о которых легла в основу понятия. При этом, чем больше объектов, тем меньше в расчете на один из них следов информаций Jk, присущих только одному или немногим объектам. И в результате согласно H=J/n при n→∞ от них ничего не остается (как это случилось с понятием материи).
Практическая польза от знания этого закона для системного анализа состоит в том, что нельзя механически переносить выводы, полученные на основе понятия одного объема, на понятие иного объема. (Многозначность вызывает логические ошибки)\\
    • Второй закон диалектики — закон развития: «Все течет и все изменяется». Это сугубо диалектический закон, поскольку в классической логике всегда A есть A и 1 есть 1. Это значит, что становление истины идет непрерывно и никогда не заканчивается, во-первых, потому что изменяется реальность, а во-вторых, совершенствуется само знание, так что никто не может претендовать на знание истины в последней инстанции.\\
    • Третий закон диалектики — закон отрицания отрицания. Пусть J есть тезис, Δ1J есть антитезис, т.е. отрицание J, а Δ2J есть антиантитезис, отрицание отрицания J. Иными словами, если J есть A, то Δ1J есть «не A», а Δ2J есть «не не A», т.е. определенный, хотя и неполный в отличие от классической логики, возврат к A. Это и есть знаменитое развитие по спирали, чреватое циклическими возвратами к изначальным формам, но с иным уже содержанием.\\
    • Четвертый диалектический закон единства и борьбы противоположностей требует избегать абсолютизации как момента борьбы, так и момента солидарности противоположностей, которые и возникают-то вследствие субъективного расчленения единого целого ради облегчения познания его противоречивых частей.\\
    • Пятый диалектический закон перехода количественных изменений в качественные акцентирует внимание на необходимости избегать абсолютизации тенденций развития, выявленных в начале процесса, ибо в дальнейшем они могут измениться вплоть до своей противоположности, причем именно вследствие развития. Формально этот закон требует учета нелинейности многих процессов, зависимости характерных показателей от его хода.
Из этого закона следует также, что сумма свойств частей не есть свойство целого, а отрицание целого не обязательно означает отрицание частей, ибо может относиться к отрицанию лишь того нового свойства, которое возникло вследствие синтеза частей.\\
    • Шестой диалектический закон всеобщей взаимосвязи и взаимозависимости явлений требует учета всех факторов, определяющих исследуемый процесс, а не только тех, что кажутся доминирующими.
Система этих шести законов является полной и замкнутой, т.е. самодостаточной для описания любых явлений. Законы диалектики имеют аналоги в классической логике, за исключением закона единства и борьбы противоположностей с его следствиями, ибо он прямо противопоставляется закону исключенного третьего. Остальные законы классической логики и диалектики различаются тем, что в первом случае любые деформации исходных суждений могут быть только скачкообразными (либо «истина», либо «ложь»), а во втором случае все переходы плавные и непрерывные с бесконечным множеством состояний между истиной и ложью. Поэтому классическая логика является бинарной, двузначной, а диалектика — бесконечно многозначна. Главная особенность диалектической логики состоит в том, что она является логикой относительной истины в отличие от классической формальной (бинарной математической) логики, являющейся логикой абсолютной истины. При всем том классическая логика является частным предельным случаем логики диалектической, когда из всех степеней истины рассматриваются только два крайних ее состояния: абсолютная истина и абсолютная ложь, которые абсолютно противопоставляются друг другу.
	Диалектическая логика — это прежде всего логика человеческого мышления на вербальном уровне, это логика слов и выражаемых ими понятий, хотя возможна ее формулировка и на образно-интуитивном (бессловесном) уровне. Классическая логика — это логика чисел.\\[Денисов]
	
	Диалектическая логика незаменима при построении модели принятия решения в условиях противоположных требований к нему.
	Исследование законов диалектики позволяет разрабатывать и исследовать модели не только в статике, но и с учетом кинематики и динамики переходных процессов в системе, а также с учетом взаимного влияния исследуемых объектов. [Волкова учебник стр 295]
	
\\Философия и теория систем: - "..категории системы, структуры, организации, части и целого (целостности) и другие принадлежат не только философии; они составляют основу, исходный пункт концептуального аппарата теории систем. Но если философские категории отображают единство и многообразие мира лишь на уровне всеобщности, то системные исследования не ограничеваются анализом систем вообще, а строят разветвленную сеть моделей систем разной степени общности, разных типов и классов организованности, а также применительно к разным предметным областям. [стр36 -37 Тюхтин] "\\
Тюхтин выделяет такую функцию теории систем, как функцию проводника: "принципы материалистической диалектики являются вместе с тем принципами системных исследований". [стр 37 Тюхтин]\\
Принципы теории систем представляют собой единство противоположностей. Диалектический принцип единства и взаимосвязи части и целого, элементов и структуры лежит в основе системных представлений. Динамическая устойчивость, надежность и гибкость высокоогранизованных систем реализуется благодаря тому, что в этих системах воплощены принципы единства дискретного и непрерывного действия (это математически обосновано Дж.фон Нейманом применительно к конечным автоматам); единства централизации в строении и функционировании систем и относительной децентрализации, автономии ее подсистем; единство разомкнутого и замкнутого контуров управления, единства прямых и обратных связей; единства последовательного и параллельного, одноканального и многоканального действия и т.д.." [стр 37 Тюхтин]\\
Все философские категории имеют две стороны (или два аспекта) - онтологическую и логико-гносеологическую. Эти стороны понятий и категорий образуют основание для построения различных предметов исследования и отличаются по характеру своих методологических функций. Онтологическая сторона, объективное содержание всех научных понятий, положений, законв, относится к объектам внешнего мира. А логико-гносеологический аспект изучения научных понятий и категорий апеллирует к самому познанию и его формам - понятиям, суждениям, выводам, теориям.[стр 38 Тюхтин]\\
Эти две стороны любых понятий и законов всегда находятся в неразрывной органической связи, в единстве.[стр 38 Тюхтин]// 


\\

\chapter*{ЗАКЛЮЧЕНИЕ }
\addcontentsline{toc}{chapter}{ЗАКЛЮЧЕНИЕ } \\

Законы диалектики образуют своеобразный понятийный каркас, позволяющий диалектически смотреть на мир, описывать его с помощью данных законов, не допуская абсолютизации каких-то процессов или явлений мира, рассмтривать последний как развивающийся объект. [Алексеев, учебник стр 126]\\
Законы диалектики не существуют оторванно  друг от друга, а реализуются как компаненты единого общего процесса развития.[Алексеев - переделано стр 126]\\
Любой предмет, явление представляет собой некоторое качество, единство его сторон, которые в результате количественного накопления противоречивых тенденций и свойств внутри этого качества приходят в противоречие, и развитие предмета осуществляется через отрицание этого качества, но с сохранением ряда свойств в образовавшемся новом качестве.[Алексеев 126]\\

Итак, основными понятиями информационного подхода являются: \\
— чувственная информация (информация восприятия), логическая и прагматическая информация, информационная сложность;\\
— для измерения сложности систем различной природы используются детерминированные и вероятностные информационные меры;\\

Параметры в формализованном представлении информационного процесса:
— параметры, характеризующие кинематику и динамику информационных процессов;\\

Компоненты формализованного представления информационного процесса:\\
— объем понятия об отображаемом объекте в формируемой модели, время восприятия, ригидность, сопротивляемость восприятию нового;\\

Информационный подход к анализу систем имеет широкий спектр приложений:\\
- позволяет получать оценку структур, свертку разнородных критериев при решении многокритериальных задач;\\
- разрабатывать методы организации сложных экспертиз;\\
- оценивать прерходные процессы принятия решений, тенденции развития систем различной физической природы и т.д. (стр.295 – 296  Волкова, В.Н. Теория систем и системный анализ: учебник для бакалавров /В.Н. Волкова, А.А. Денисов. - М. : Издательство Юрайт, 2012. - 679 с. - Серия: Бакалавр. Углубленный курс.)\\


Таким образом, изложенный в данной работе информационный подход позволяет формализовать описание информационного взаимодействия субъектов в терминах и понятиях «информационного поля». Это делает возможным в дальнейшем переход к количественному описанию информационных процессов в обществе, к разработке методик контроля и управления этими процессами. \\
Информационный подход применяется для сравнительного анализа иерархических структур, для макроэкономического моделироваия и  имеет свое приложение к моделированию процессов принятия решений.\\
На примере рассмотренного в данной работе информационного подхода в системном анализе, можно сделать вывод о том, что…. философия как наука может выполнять по отношению к теории систем и системному анализу логико-гносеологическую функцию, выполняя таким образом методологическую функцию.  \\
Воздействие философии на построение отдельных теорий может быть не интегрально, а фрагментарно, локально, ...конкретная наука может опираться на отдельные философские принципы, идеи и понятия. \\


Любой метод науки имеет свои теоретико-познавательные и логические возможности, за пределами которых его эффективность снижается или элиминируется. В следствие этого для изучения сколько-нибудь сложных объектов требуются комплексы методов, способные компенсировать неполноту познавательных возможностей отдельных методов. Развитие информационного подхода позволяет надеяться на то, что с помощью этого подхода можно глубже понять и исследовать наиболее высокоорганизованные сферы природы — социально-экономическую, а возможно — и даже трансцендентную сферу.\\

При моделировании наиболее сложных процессов, таких как целеобразование, совершенствование организационных структур и т.п., принцип развития (как диалектическая категория) может быть реализован в форме методики системного анализа. \\
решение проблем сложности и нелинейности применительно к практической деятельности в вопросах организации и управления информационными, экономическими, социальными и др. системами - это поле для междисциплинарных исследований, поле поиска подходов и методов, востребованных как в науке, так и на практике. \\

Диалектика вместе с формальной логикой в информационном подходе призвана обеспечить достижение  адекватности. «По мере усложнения задач получение модели и доказательство ее адекватности усложняется.» (стр 10 Денисов) Диалектика разрабатывает средства наиболее полного, точного отражения развивающейся, непрерывно изменяющейся сущности объекта. 
 Диалектическая концепция объясняет объективность противоречий и даже необходимость их для существования и развития сложного мира.\\

Информационный подход (рассматриваемый в гл. 3) развивает диалектическую концепцию естествознания, ставшую по мнению философов, занимающихся исследованием концепций естествознания, приоритетной в ХХ в. При этом теория представляет процессы восприятия информации, проявление законов диалектики в формализованном виде, более приемлемом европейской культурой. Развитие информационного подхода позволяет надеяться на то, что с помощью этого подхода можно глубже понять и исследовать наиболее высокоорганизованные сферы природы — социально-экономическую, а возможно — и даже трансцендентную сферу.\\

Концепция информационного поля позволяет найти количественную оценку содержания, смысла на основе прослеживания путей реализации логических связей. При этом «содержание» выступает как «смысл» взаимодействия неживых объектов в соответствии с «целями» законов природы.\\
...для практических приложений оказался удобнее дискретный вариант теории , с помощью которого пояснял идею отражения материи более популярно для инженеров и который позволил ввести меры информации .
Универсальное моделирование – это философкие категории, модель мира, модель познания, -теория отражение как универсальное моделирование\\

	\chapter*{СПИСОК ИСПОЛЬЗОВАННЫХ ИСТОЧНИКОВ} 
\addcontentsline{toc}{chapter}{СПИСОК ИСПОЛЬЗОВАННЫХ ИСТОЧНИКОВ}\\
1. Виноградов С.Н., Кузьмин А.Ф. Логика: учебник для средней шкрлы – М.: Издательство “Наше завтра”, 2022. -176 с.\\
2. Вдовин В. М. Теория систем и системный анализ: Учебник для бакалавров / В. М. Вдовин, Л. Е. Суркова, В. А. Валентинов. — 3-е изд. — М.: Издательско-торговая корпорация «Дашков и К°», 2016. — 644 с.\\
3. Волкова, В.Н. Теория систем и системный анализ: учебник для бакалавров /В.Н. Волкова, А.А. Денисов. - М. : Издательство Юрайт, 2012. - 679 с. - Серия: Бакалавр. Углубленный курс.\\
4. Введение в философию: Учебник для вузов. В 2ч. Ч. 2./ Фролов И.Т., Араб-Оглы Э.А., Арефьева Г.С. и др. - М.: Политиздат, 1989. - 639 с.\\
5. Волкова В.Н. Теоретические основы информационных систем / В.Н. Волкова. − СПб.: Изд-во Политехн. ун-та, 2012. − 280 с.\\
6. Козлов В.Н. Теория информационного поля и информационный подход к анализу систем./ Прикладная информатика, №3 (27) 2010.-\\
7. Современные проблемы системного анализа: Изд. 3-е, перераб.  и дополн. / Денисов А.А. – СПб.: Изд-во Политехн. ун-та, 2009. с.\\
8. Денисов А.А. Современные проблемы системного анализа: Информационные основы . – СПб .: Изд - во Политехн . ун - та , 2005. − 296 с \\
9. Теория систем и системный анализ: учебное пособие / Н.А. Сергеева. - Красноярск: Сиб. федер. ун-т, 2013. - 110 с.\\
    
    
    отправить на oikonnikov@sfu-kras.ru
   	\end{document} 